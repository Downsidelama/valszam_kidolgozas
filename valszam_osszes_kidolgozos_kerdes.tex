\documentclass[12p]{article}
\setlength{\parindent}{0em}
\usepackage{amsmath}
\usepackage[utf8]{inputenc}
\usepackage[english]{babel}
\renewcommand{\baselinestretch}{1.5}

\begin{document}

(1) Adjon formulát abszolút folytonos eloszlású X valószínűségi változó esetén az Y = g(X) valószínűségi változó várható értékére!

(2) Legyen X standard normális eloszlású valószínűségi változó. Számítsa ki az X várható értékét! 

(3) Mondja ki Bayes tételét. (3x)

(4) Igazolja, hogy független azonos paraméterű binomiális eloszlású valószínűségi változók összege is ilyen

(5) Írja fel az X és Y együttes eloszlásfüggvénye és az X perem eloszlásfüggvénye között fennálló kapcsolatot

(6)  Definiálja a szórásnégyzetet és mondja ki legfontosabb tulajdonságait. (3x)

(7) Definiálja a konfidencia-intervallum fogalmát (3x)

(8) Definiálja az első- és a másodfajú hiba fogalmát

(9) Adja meg a lineáris regresszió feladatát és megoldását

(10) Definiálja idősorok erős stacionaritását

(11) Definiálja az ASN fogalmát.

(12) írja fel az eloszás- és a sűrűségfüggvényeket karakterizáló tulajdonságokat! (3x)

(13) Legyen X pascal eloszlású valószínűségi változó. Számítsa ki az X várható értékét!

(14) Mi a kapcsolat az X és Y valószínűségi változó függetlensége és együttes eloszlásfüggvénye között? (3x)

(15) írja fel, az X diszkrét valószínűségi változó várható értkének definícióját! (2x)

(16) Mondja ki Bayes tételét. (3x)

(17) Vezesse le a polinomiális eloszlás koordinátái közötti korrelációra vonatkozó képletet.

(18) Definiálja a Poisson folyamatot. (3x)

(19) Mondja ki a nagy számok törvényének minél több változatát. (3x)

(20) Vezesse le a Poisson eloszlás paraméterére a maximum likelihood becslést. (3x)

(21) Definiálja a lineáris modell hipotéziseit és adja meg a próbastatisztikát.

(22) Legyen X +1 és -1 értéket felvevő valószínűségi változó, $P(X = 1) = 1/6$ és $P(X = -1) = 5/6$. Számítsuk ki X várható értékét és szórás négyzetét.

(23) Bizonyítsa be a nagy számok gyenge törvényét. (2x)

(24) Legyen $X^2$ egyenletes a $[0, 1]$-en. Adja meg X eloszlását és várható értékét.

(25) Irja fel a rendezett minta k-adik elemének a sűrűségfüggvényét.

(26) Rajzolja fel, hogyan lehet a sűrűségfüggvények segítségével a próbák kritikus értékeit és erőfüggvényüket szemléltetni!

(27) Definiálja a $\chi$-négyzet próbát és adja meg legfontosabb alkalmazásait.

(28) Adja meg a lineáris regresszió feladatát és alkalmazásait.

(29) Vezesse le a normális eloszlás várható értékére a momentum módszerrel adódó becslést.

(30) Adjon módszert idősorok simítására.

(31) Definiálja a szórásnégyzetet és mondja ki legfontosabb tulajdonságait.

(32) Mondja ki a centrális határeloszlás tételt. (2x)

(33) Legyen $X$ olyan valószínűségi változó, amely a (0, 1) intervallumból veszi fel az értékeit. Eloszlásfüggvénye ott F(t) = t/2 ha $t \in (0, 1/2]$, és $F(t) = 3/4 + t/4$ ha $t \in (1/2, 1]$. Rajzoljuk fel X eloszlásfüggvényét és számítsuk ki a $P(X \leq 1/2)$, valamint a $P(1/4 \leq X \leq 3/4)$ és $P(X = 1/2)$ valószínűségeket!

(34) Definiálja a teljes eseményrendszer fogalmát! (2x)

(35) Definiálja a folytonos eloszlásokra a várható érték fogalmát.

(36) Mondja ki Bernstein tételét.

(37) Bizonyítsa be a Markov egyenlőtlenség általános alakját.

(38) Definiálja a maximum-likelihood módszert.

(39) Definiálja a Wilcoxon próbált.

(40) Definiálja az autoregressziós és a mozgóátlag folyamatokat.

(41) Mit jelent a becsléses illeszkedésvizsgálat és hogyan alkalmazzuk rá a $\chi$-négyzet próbát?

(42) Legyen X n-edrendű p paraméterű binomiális eloszlású valószínűségi változó. Számítsa ki az X várható értékét! 

(43) Mondja ki a szita (Poincaré) formulát

(44) Definiálja a kovarianciát és írja le tulajdonságait. (4x)

(45) Mondja ki a centrális határeloszlás tétel lokális változatát. (2x)

(46) Legyen X egyenletes a [0,1]-en. Adja meg $X^2$ eloszlását és várható értékét.

(47) Irja fel a Nadarajah módszer lényegét és szemléltesse grafikusan az eljárás motivációját! 

(48) Definiálja a 2 mintás t-próbát és vezesse le a próbastatisztika képletét! 

(49) Mondja ki a Neyman-Pearson lemmát!

(50) Definiálja az autokorrelációs együtthatókat és adja meg becslésüket arra az esetre, amikor reziduálisokból számolunk! 

(51) Definiálja a valószínűségi mező fogalmát! (2x)

(52) Vezesse le, hogy a korrigált tapasztalati szórásnégyzet torzítatlan becslés a szórásnégyzetre. 

(53) Vezesse le, hogy a relatív gyakoriság konzisztens becslés a valószínűségre.

(54) Definiálja becslések aszimptotikus torzítatlanságát.

(55) Definiálja a t-próbát az egy- és kétmintás esetre is.

(56) Definiálja a szekvenciális próbát (egyszerű hipotézisek esetére).

(57) Definiálja a korrelációs együtthatót és mondja ki legfontosabb tulajdonságait.

(58) Mikor nevezünk egy torzítatlan becslést hatásosabbnak egy másiknál?

(59) Definiálja idősorokra a trend és a periódus fogalmát.

(60) Adjon példát olyan maximum likelihood becslésre, ami nem torzítatlan! (3x)

(61) Mondja ki abszolút folytonos valószínűségi változók függetlenségének ekvivalens jellemzőit.

(62) Definiálja valószínűségi változók eloszlását

(63) Mit jelent a valószínűség folytonossága?

(64) Definiálja n valószínűségi változó függetlenségét.

(65) Definiálja diszkrét valószínűségi változók együttes eloszlását!

(66) Definiálja a t eloszlást!

(67) Definiálja a gyenge konvergenciát!

(68) Irja fel annak a valószínűségét, hogy a 90/5-ös lottónál egy szelvénnyel játszva legalább 2 találatunk lesz!

(69) Legyen az alábbi 3 megfigyelésünk az $(X, Y)$ párra. $(1;2)$ $(1.2;2.5)$ $(0.8;1.6)$. Nadarajah
módszerét használva a $k(x) = 1$, ha $-1/2 < x < 1/2$ (és 0 különben) magfüggvénnyel és a $h = 0.5$ ablakszélességel, mi lesz az $E(Y|X = 1.1)$ becslése?

(70) Vezesse le egyszerű hipotézisekre, hogy az egymintás u-próba valószínűséghányadospróba az egyoldali ellenhipotézis esetén!

(71) Hogyan tudjuk a lineáris modellt alkalmazni polinommal történő közelítésre?

(72) Adjon példát olyan maximum likelihood becslésre, mely nem torzítatlan.

(73) Definiálja az előjelpróbát! Mik a kritikus értékek? (3x)

(74) Adjon példát olyan $X_n$ valószínűségi változó-sorozatra, amely független tagú, de mégsem teljesül rá a centrális határeloszlástétel.

(75) Legyen X egyenletes eloszlású a [-1, 1] intervallumon. Számítsa ki $X^2$ sűrűségfüggvényét.

(76) Definiálja a normális eloszlást és vezesse le a legfontosabb tulajdonságait.

(77) Mondja ki és bizonyítsa is be a Markov egyenlotlegnséget.

(78) Hogyan alkalmazhatjuk a Kolmogorov-Szmirnov próbát véletlenszám-generátorok tesztelésére?

(79) Definiálja a binomiális eloszlást, vezesse le a képletét és adja meg alkalmazási lehetőségeit!

(80) Legyenek az adataink: 1,2,5,6. Számolja ki a mediánt és rajzolja fel a tapasztalati eloszlásfüggvényt!

(81) Tegyük fel, hogy egy 1ö fős csoportban mindenki kihúzza valakinek a nevét a karácsonyi ajándékozásnál. várható értékben hány ember húzta saját magát? Vezesse is le az eredményt!

(82) Rajzolja fel a $\lambda = 1$ és $\lambda = 2$ paraméterű exponenciális eloszlás sűrűség- és eloszlásfüggvényét!

(83) Vezesse le a $[0, \theta]$ intervallumon egyenletes eloszlásra a maximum likelihood becslést n mintaelem alapján!

(84) Rajzolja fel az alábbi minta tapasztalati eloszlásfüggvényét: 1,5,6,7! (2x)

(85) Hogyan tudunk inverz módszerrel véletlen számot generálni? (2x)

(86) Legyen az X valószínűségi változó sűrűségfüggvénye $f(x) = x^2/9$, ha $0 < x < 3$ és $0$ különben. Adjuk meg $X$ eloszlásfüggvényének értékét az $1/2$ helyen. $E(1/X)=?$ Adjuk meg $sqrt{X}$ sűrűségfüggvényének értékét az $1/2$ helyen.

(87) Számítsa ki annak a valószínűségét, hogy egy háromgyerekes családban azonos neműek a gyerekek! (Tegyük fel, hogy a fiúk és lányok szóletési valószínűsége is 0.5 és hogy az egyes születések függetlenek.)

(88) Definiálja a várható érték fogalmát diszkrét valószínűségi változókra!

(89)  Legyen X egyenletes eloszlású a $[-2,2]$ intervallumon. Adja meg $(X - 2)(X + 2)$ sűrűségfüggvényét.

(90) Mondja ki abszolút folytonos valószínűségi változók függetlenségének ekvivalens
jellemzőit.

(91) Mondjon példát az egyenletes, az exponenciális és a normális eloszlás alkalmazására!

(92) Definiálja a bootstrap módszert.

(93) Mi a leíró statisztika feladata?

(94) Definiálja a kovarianciát!

(95) Mihez tart $P(X1+X2+ ... +Xn > 9n/2)$, ha $X1, X2, . . .$ független, a $[4,6]$ intervallumon
egyenletes eloszlású valószínűségi változók és n tart végtelenhez?

(96) Vezesse le az exponenciális eloszlás várható értékére vonatkozó képletet!

(97) Legyen az alábbi 3 megfigyelésünk az $(X, Y)$ párra. $(1;2)$ $(1.3;2.5)$ $(0.8;1.5)$. Nadarajah
módszerét használva a a $k(x) = 1$, ha $-1/2 < x < 1/2$ (és 0 különben) magfüggvénnyel és a $h = 0.5$ ablakszélességgel, adjon becslést $Y$ közelítésére, ha $X = 1$.

(98) Definiálja az egy-és kétmintás u-próbát.

(99) Az alábbi minta: 0,0,1,1,2 alapján becsüljük torzítatlanul az alábbi paramétereket:
várható érték, szórásnégyzet, annak a valószínűsége, hogy $X > 0.2$.

(100) Adjon módszert arra, hogyan szimulálna véletlen számot a $\sin x (0 < x < \pi/2)$ sűrűségfüggvényű eloszlásból!

(101) Hogy hívjuk a $h(t) = E(|X - t|^2)$ függvény minimum helyét?

(102) Hogyan adhatjuk meg a megszámlálható valószínűségi mezőn a valószínűséget?

(103) Legyen X 5 várható értékű és 2 szórású normális eloszlású, Y pedig az X = 7 esemény indikátora. Számítsa ki Y várható értékét és szórását!

(104)  Definiálja a visszatevés nélküli mintavétel modelljét és adja meg a különböző selejtszámok valószínűségeit!

(105) Mondja ki a Markov egyenlőtlenséget!

(106) Vezesse le a valószínűségre a maximum likelihood becslést n mintaelem alapján!

(107) Definiálja a sztochasztikus konvergenciát! (2x)

(108) Írja le a sűrűségfüggvény becslésére tanult eljárást.

(109) Adja meg a legkisebb négyzetes becslést a lineáris modellben. Mik a tulajdonságai?

(110) Legyen X egyenletes a $[0, 1]$-en. Adja meg $X^2$ eloszlását és várható értékét.

(111) Számolja ki a Wilcoxon próbastatisztika értékét az alábbi két adatsorra! Hogyan döntene arról, hogy tekinthetők-e azonos eloszlásból származónak? Az első minta: 1,2,4,12,23,50;
a második minta: 3,8,15,25,61,200.

(112) Definiálja az F-próbát és adja meg legfontosabb alkalmazásait. (2x)

(113)  Hogyan generálhatunk véletlen számot Neumann módszerével?

(114) Töltse ki az alábbi táblázat hiányzó celláit úgy, hogy a peremeloszlások egyenletesek
legyenek a megadott számokon! Függetlenek-e ezek a változók? Számolja ki az $E(X^2Y)$ várható értéket!

\begin{center}
 \begin{tabular}{| c c || c | c | c | c |} 
 \hline
 Y & X &  0 & 1 & 2 & 3\\ [0.5ex] 
 \hline\hline
 0 &  & 0,2 & & 0,05 & \\ 
 \hline
 1 &  & & 0,15 & & \\
 \hline\hline
\end{tabular}
\end{center}

(115) Mutassa meg, hogy független azonos paraméterű binomiálisok összege binomiális eloszlású! Mi a kapcsolat a rendek között?

(116) Definiálja két esemény függetlenségét!

(117)  Számítsa ki két kockadobás maximumának a várható értékét.

(118) Definiálja az első- és másodfajú hiba fogalmát, valamint próbák konzisztenciáját.

(119) Irja le a t-próba alkalmazását a párosított megfigyelések esetére.

(120) Vajon van-e stacionárius megoldása az $X_n = 0.96 X_{n-1} + \epsilon_n$ rekurziónak? Válaszát
indokolja!

(121) Legyen a mintaelemek sűrűségfüggvénye $f(x) = (\theta + 1)x^{\theta}$ ha $0 < x < 1$ ($\theta \geq$ 0 a par.).
Egy mintaelem alapján határozzuk meg az $\alpha = 0.05$ terjedelmű valószínűséghányadospróbát a $H_0 : \theta = 0, H_1 : \theta = 1$ hipotézisekre!

(122) Ismertesse a lineáris regresszió együtthatóira vonatkozó próbák lényegét!

(123) Adott n elemű minta az alábbi eloszlásból: $P(Z = 1) = c$, $P(Z = 2) = 2c$, $P(Z = 3) = 1 - 3c$ ($0 < c < 1/3$ az ismeretlen paraméter). Határozza meg a paraméter maximum
likelihood becslését!

(124) Számítsa ki két kockadobás minimumának a várható értékét.

(125) Írja fel a Csebisev egyenlőtlenséget, és vázolja a bizonyítását.

(126) Legyen X 5 várható értékű Poisson eloszlású, Y pedig az X = 5 esemény indikátora. Számítsa ki Y várható értékét és szórását!

(127) Adjon példát olyan maximum likelihood becslésre, mely nem torzítatlan

(128) Adja meg a Wilcoxon-próba hipotéziseit, a próbastatisztikát és a kritikus tartományt.
Mikor alkalmazná ezt a próbát? (2x)

(129) Adjon meg két módszert integrálok szimulációval történő kiszámítására! Hasonlítsa is
össze őket!

(130) Definiálja a 2 mintás t-próbát és vezesse le a próbastatisztika képletét!

(131)  Ismertesse a lineáris regresszió együtthatóira vonatkozó próbák lényegét!

(132) Vezesse le a normális eloszlás várható értékére a maximum likelihood becslést n
mintaelem alapján!

(133) Írja fel, az X diszkrét valószínűségi változó várható értékének definícióját és sorolja fel
a várható érték legfontosabb tulajdonságait!

(134) Legyen $\Omega = {1, 2, 3, 4, 5, 6}$, $P(i) = 1/6 (1 \leq i \leq 6)$ és $A = {1, 2, 3}$. Adon meg olyan B
eseményt, amire $0 < P(B) < 1$ és B és A függetlenek! Válaszát indokolja

(135) Tegyük fel, hogy az X valószínűségi változó sűrűségfüggvénye $f(x) = 2(1-x)$ ha $0 < x < 1$ és 0 különben. E(X) =? 

(136) Legyen az X valószínűségi változó eloszlása $P(X = k) = (5 - k)/15$, $ha 0 \leq k \leq 4$. $D^2(X)$ = ?

(137) Mondja ki a nagy számok Bernoulli-féle törvényét!

(138) Definiálja egy valószínűségi változó mediánját.

(139) Mi a lényeges különbség a leíró és a matematikai statisztika között?

(140) Vezesse le a maximum likelihood becslést a binomiális eloszlás p paraméterére (az n paramétert tekintsük ismertnek)!

(141) Definiálja a hipotézisvizsgálat alapfogalmait!

(142)  Definiálja az u-próbát az egy- és kétmintás esetre is.

(143) Mi a hasonlóság és mi a különbség az előjel- és a Wilcoxon próba között?

(144) Adjon módszert arra, hogyan szimulálna véletlen számot a sin(x) $(0 < x < \pi/2)$ sűrűségfüggvényű eloszlásból!

(145) Definiálja a korrelogrammot és adjon módszert, hogyan lehet ennek segítségével tesztelni
az autokorrelációk szignifikanciáját.

(146) Definiálja valószínűségi változók általános fogalmát.

(147) Definiálja n darab esemény függetlenségét.

(148) Mutassa meg, hogy a peremeloszlások nem határozzák meg az együttest!

(149) Adjon példát korrelálatlan de nem független valószínűségi változó-párra!

(150) Legyen az X eloszlása a következő: $P(X = k) = k/6$, ha $k = 1, 2, 3$. Számolja ki $E(X)$ és $E(X^2)$ értékét!

(151) Mondja ki a centrális határeloszlás tételt! Hol használtuk a statisztikában?

(152) Hasonlítsa össze a várható érték tesztelésre vonatkozó próbákat!

(153) Tegyük fel, hogy a mintánk megfigyelt értékei: 0, 0, 2, 2 és azt is, hogy közelíthetőek 1 szórású normális eloszlással. Írjon fel 0.95 megbízhatóságú konfidencia intervallumot a
várható értékre!

(154) Definiálja a $\chi$-négyzet próbát és adja meg legfontosabb alkalmazásait

(155) Egy faktor hatását vizsgáljuk. Az alacsony szinten 4 mérésünk volt: 8, 7, 11, 10. A felső
szinten az eredmények: 5,7,3,9. Hogyan vizsgálnánk a hatás szignifikanciáját? Adjuk meg
a próbastatisztika értékét.

(156) Melyik konstans közelíti az $X$ változót a legkisebb négyzetes hibával? Válaszát indokolja

(157)  Irja le a Monte Carlo módszerek lényegét és adjon példát alkalmazásukra

(158) Definiálja a gyengén stacionárius idősorokat és adjon rá egy példát!

(159) Legyen az X valószínűségi változó sűrűségfüggvénye $f(x) = 2(3 - x)/9$ ha $0 < x < 3$ (és 0 különben). $E(X)$ =?, $D^2(X)$ =? 

(160) Legyen X a pikkek, Y pedig a királyok száma annál a kísérletnél, ahol kétszer húzunk
visszatevéssel az 52 lapos francia kártyacsomagból (ebben 4 szín van és minden színből 13
lap). Írja fel X, Y együttes eloszlását!

(161) Definiálja a feltételes valószínűséget!

(162) Mondja ki a teljes valószínűség tételét!

(163) Irja le a Parzen-Rosenblatt módszert és tulajdonságait.

(164) Adja meg valószínűségi változók függetlenségének karakterizációit!

(165) Definiálja a QQ plot diagramot! Mikor használjuk?

(166) Mit jelent a becsléses illeszkedésvizsgálat és hogyan alkalmazzuk rá a $\chi$-négyzet próbát?

(167) Milyen modern módszerekkel tudjuk a statisztikai eljárásokat a big data esetére alkalmazni?

(168) Irja le a tanult, Neumann-féle módszert véletlen számok generálására!

(169) Mi a szóráselemzés lényege?







\end{document}