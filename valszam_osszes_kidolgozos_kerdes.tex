\documentclass[12p]{article}
\setlength{\parindent}{0em}
\usepackage{amsmath, mathrsfs, listings}
\usepackage{upquote, amssymb}
\usepackage[utf8]{inputenc}
\usepackage[english]{babel}
\usepackage[%
    left=0.50in,%
    right=0.50in,%
    top=1.0in,%
    bottom=1.0in%
]{geometry}%
\renewcommand{\baselinestretch}{1.5}

\begin{document}

(1) Adjon formulát abszolút folytonos eloszlású X valószínűségi változó esetén az Y = g(X) valószínűségi változó várható értékére!

(2) Legyen X standard normális eloszlású valószínűségi változó. Számítsa ki az X várható értékét! 

(3) Mondja ki Bayes tételét. (3x)

Legyen B1, B2, ..., pozitív valószínűségű eseményekből álló teljes eseményrendszer, A $\in$ $\mathscr{A}$ pozitív valószínűségű. Ekkor

$$\displaystyle{P(B_k|A) = \frac{P(A|B_k)P(B_k)}{\displaystyle{\sum_i} P(A|B_i)P(B_i)}}$$

(4) Igazolja, hogy független azonos paraméterű binomiális eloszlású valószínűségi változók összege is ilyen.

(5) Írja fel az X és Y együttes eloszlásfüggvénye és az X perem eloszlásfüggvénye között fennálló kapcsolatot.

(6)  Definiálja a szórásnégyzetet és mondja ki legfontosabb tulajdonságait. (3x)

$$D^2X = E[(X - EX)]^2 = EX^2 - E^2X$$

Tulajdonságok:

\begin{itemize}
	\item $\sigma \geq 0$
	\item $\sigma = 0$ csak akkor teljesül, ha minden ismérvérték egyenlő.
	\item Ha minden ismérvértékhez hozzáadjuk ugyanazt a valós számot, a szórás változatlan marad.
	\item Ha minden ismérvértéket megszorozzuk ugyanazzal az a valós számmal, a szórás az a szám abszolútértékével szorzódik.
\end{itemize}

(7) Definiálja a konfidencia-intervallum fogalmát (3x)

A konfidenciaintervallum a valószínűségi intervallum, az induktív statisztika eszköze: ha mintából becsülünk, sosem tudjuk a pontos értéket, a teljes sokaság felmérése igen drága dolog. A konfidenciaintervallum adott szignifikanciaszinten: a becsült változó alsó és felső korlátja.

(8) Definiálja az első- és a másodfajú hiba fogalmát

(9) Adja meg a lineáris regresszió feladatát és megoldását

(10) Definiálja idősorok erős stacionaritását

(11) Definiálja az ASN fogalmát.

(12) írja fel az eloszás- és a sűrűségfüggvényeket karakterizáló tulajdonságokat! (3x)

Eloszlásfüggvény:\\
$F_X(z):=P(X<z)$. $F_X(z):\mathbb{R}\rightarrow\mathbb{R}$ függvény az X valószínűségi változó eloszlásfüggvénye.\\
Tulajdonságai:

\begin{itemize}
	\item $0 \leq F_X(z) \leq 1$
	\item $F_X(z)$ monoton nő
	\item $\lim_{z \rightarrow \infty} F_X(z) = 1; \lim_{z \rightarrow -\infty} F_X(z) = 0$
	\item $F_X(z)$ balról folytonos
\end{itemize}

Sűrűségfüggvény:\\
Az X valószínűségi változó sűrűségfüggvénye $f(x) \leftrightarrow X$-re:

$F_X(z) = \int_{-\infty}^z f(t) dt$

Tulajdonságai:
\begin{itemize}
	\item $f \geq 0$
	\item $\int_{-\infty}^{\infty} f(t) dt = 1$
	\item Minden ilyen f integrálfüggvénye eloszlásfüggvény
\end{itemize}

(13) Legyen X pascal eloszlású valószínűségi változó. Számítsa ki az X várható értékét!

(14) Mi a kapcsolat az X és Y valószínűségi változó függetlensége és együttes eloszlásfüggvénye között? (3x)

(15) írja fel, az X diszkrét valószínűségi változó várható értkének definícióját! (2x)

(16) -

(17) Vezesse le a polinomiális eloszlás koordinátái közötti korrelációra vonatkozó képletet.

(18) Definiálja a Poisson folyamatot. (3x)

Poisson folyamat: időben lejátszódó folyamatnál adott [a,b) intervallumba eső események száma $(X_{a,b})$ éppen $\lambda(b-a)$ paraméterő Poisson eloszlású, ha a folyamat

\begin{itemize}
	\item homogén: $X_{a,a+t}$ eloszlása csak t-tıl függ;
	\item utóhatás nélküli: $X_{a,b}$ és $X_{b,c}$ függetlenek ha a<b<c;
	\item nemelfajuló: $0<P (X_{a,b}=0)<1$.

\end{itemize}

(19) Mondja ki a nagy számok törvényének minél több változatát. (3x)

(20) Vezesse le a Poisson eloszlás paraméterére a maximum likelihood becslést. (3x)

Zyxon megoldasas:

A likelihood fuggveny:

$$L(\theta;\underline{x}) = f_{\theta}(\underline{x}) = \prod^n_{i=1} f_{\theta}(x_i)$$

Ennek maximumhelye lesz a $\theta$ paraméter maximum likelihood becslése. A Poisson-eloszlás paramétere: $\overline{X}$

Tankonyvar.hu-s megoldas:

$$P(X_1 = k) = \frac{\lambda^k}{k!}e^{-\lambda}\quad k=0, 1, 2, ...$$
A likelihood-fuggveny
$$L(x_1, ..., x_n;\lambda)=\prod^n_{i=1}\frac{\lambda^{x_i}}{x_i!}e^{-\lambda}$$
A loglikelihood-fuggveny
$$l(x_1, ..., x_n;\lambda) = \sum^n_{i=1}(x_i log \lambda - \lambda - log x_i!)$$
A maximumhelyet derivalassal hatarozzuk meg:
$$0 = \frac{\theta l(x_1, ..., x_n;\lambda)}{\theta \lambda} = \frac{1}{\lambda} \sum^n_{i=1} x_i - n$$
Innen
$$\hat{\lambda} = \frac{1}{n} \sum^n_{i=1} X_i = \overline{X}$$
A maximum-likelihood becsles.

(21) Definiálja a lineáris modell hipotéziseit és adja meg a próbastatisztikát.

(22) Legyen X +1 és -1 értéket felvevő valószínűségi változó, $P(X = 1) = 1/6$ és $P(X = -1) = 5/6$. Számítsuk ki X várható értékét és szórás négyzetét.

(23) Bizonyítsa be a nagy számok gyenge törvényét. (2x)

(24) Legyen $X^2$ egyenletes a $[0, 1]$-en. Adja meg X eloszlását és várható értékét.

(25) Irja fel a rendezett minta k-adik elemének a sűrűségfüggvényét.

(26) Rajzolja fel, hogyan lehet a sűrűségfüggvények segítségével a próbák kritikus értékeit és erőfüggvényüket szemléltetni!

(27) Definiálja a $\chi$-négyzet próbát és adja meg legfontosabb alkalmazásait.

(28) Adja meg a lineáris regresszió feladatát és alkalmazásait.

(29) Vezesse le a normális eloszlás várható értékére a momentum módszerrel adódó becslést.

(30) Adjon módszert idősorok simítására.

(31) Definiálja a szórásnégyzetet és mondja ki legfontosabb tulajdonságait.

(32) Mondja ki a centrális határeloszlás tételt. (2x)

151.-es kerdesben benne van a valasz.

(33) Legyen $X$ olyan valószínűségi változó, amely a (0, 1) intervallumból veszi fel az értékeit. Eloszlásfüggvénye ott F(t) = t/2 ha $t \in (0, 1/2]$, és $F(t) = 3/4 + t/4$ ha $t \in (1/2, 1]$. Rajzoljuk fel X eloszlásfüggvényét és számítsuk ki a $P(X \leq 1/2)$, valamint a $P(1/4 \leq X \leq 3/4)$ és $P(X = 1/2)$ valószínűségeket!

(34) Definiálja a teljes eseményrendszer fogalmát! (2x)

(35) Definiálja a folytonos eloszlásokra a várható érték fogalmát.

(36) Mondja ki Bernstein tételét.

(37) Bizonyítsa be a Markov egyenlőtlenség általános alakját.

(38) Definiálja a maximum-likelihood módszert.

(39) Definiálja a Wilcoxon próbált.

(40) Definiálja az autoregressziós és a mozgóátlag folyamatokat.

(41) Mit jelent a becsléses illeszkedésvizsgálat és hogyan alkalmazzuk rá a $\chi$-négyzet próbát?

(42) Legyen X n-edrendű p paraméterű binomiális eloszlású valószínűségi változó. Számítsa ki az X várható értékét! 

(43) Mondja ki a szita (Poincaré) formulát

(44) Definiálja a kovarianciát és írja le tulajdonságait. (4x)

$cov(X,Y) := E[(X-E(X))(Y-E(Y))]$

Kiszámítása:
$cov(X,Y) = E(XY)-E(X)E(Y)$

Ha X és Y függetlenek $\rightarrow$ $cov(X,Y) = 0$, fordítva nem feltétlenül igaz

A kovariancia szimmetrikus: $cov(X,Y) = cov(Y,X)$

$cov(X,X) = D^2(X)$

(45) Mondja ki a centrális határeloszlás tétel lokális változatát. (2x)

Legyenek $X_1 , X_2 ,..., X_n ,...$ független, azonos eloszlású, abszolút folytonos valószínőségi változók. Tegyük fel, hogy $\sigma_2=D_2(X)$ véges $(m:=E(Xi))$ és hogy $f(x)$ korlátos. Tekintsük a standardizált összegüket:

$$Z_n := \frac{X_1 + ... + X_n - nm}{\sqrt{n}\sigma}$$

Ekkor $Z_n$ sűrűségfüggvénye konvergál a standard normális
eloszlás sőrőségfüggvényéhez, azaz

$$\lim_{n \rightarrow \infty} f_n(x) = \frac{1}{\sqrt{2\pi}} e^{- \frac{x^2}{2}}$$

A konvergencia z-ben egyenletes.

(46) Legyen X egyenletes a [0,1]-en. Adja meg $X^2$ eloszlását és várható értékét.

(47) Irja fel a Nadarajah módszer lényegét és szemléltesse grafikusan az eljárás motivációját! 

(48) Definiálja a 2 mintás t-próbát és vezesse le a próbastatisztika képletét! 

(49) Mondja ki a Neyman-Pearson lemmát!

(50) Definiálja az autokorrelációs együtthatókat és adja meg becslésüket arra az esetre, amikor reziduálisokból számolunk! 

(51) Definiálja a valószínűségi mező fogalmát! (2x)



(52) Vezesse le, hogy a korrigált tapasztalati szórásnégyzet torzítatlan becslés a szórásnégyzetre. 

(53) Vezesse le, hogy a relatív gyakoriság konzisztens becslés a valószínűségre.

(54) Definiálja becslések aszimptotikus torzítatlanságát.

(55) Definiálja a t-próbát az egy- és kétmintás esetre is.

(56) Definiálja a szekvenciális próbát (egyszerű hipotézisek esetére).

(57) Definiálja a korrelációs együtthatót és mondja ki legfontosabb tulajdonságait.

(58) Mikor nevezünk egy torzítatlan becslést hatásosabbnak egy másiknál?

(59) Definiálja idősorokra a trend és a periódus fogalmát.

(60) Adjon példát olyan maximum likelihood becslésre, ami nem torzítatlan! (3x)

(61) Mondja ki abszolút folytonos valószínűségi változók függetlenségének ekvivalens jellemzőit.

(62) Definiálja valószínűségi változók eloszlását.

(63) Mit jelent a valószínűség folytonossága?

(64) Definiálja n valószínűségi változó függetlenségét.

 Az $X_1,...,X_n$ valószínűségi változók függetlenek, ha az $\mathscr{F}_{X1}, \mathscr{F}_{X2} ,...,\mathscr{F}_{X_n}$ generált $\sigma$-algebrák függetlenek.

(65) Definiálja diszkrét valószínűségi változók együttes eloszlását!

az X diszkrét valószínűségi változó, ha értékkészlete $(x_1 ,..., x_n...)$ legfeljebb megszámlálható.\\
A valószínűségi változó definíciójából adódóan $\{\omega:X(\omega)= x_i\}=\{X=x_i\}= \in \mathscr{A}$ azaz $p_i:=P$ $(X=x_i)$értelmes. Ezek meg is határozzák X eloszlását.

(66) Definiálja a t eloszlást!

(67) Definiálja a gyenge konvergenciát!

(68) Irja fel annak a valószínűségét, hogy a 90/5-ös lottónál egy szelvénnyel játszva legalább 2 találatunk lesz!

(69) Legyen az alábbi 3 megfigyelésünk az $(X, Y)$ párra. $(1;2)$ $(1.2;2.5)$ $(0.8;1.6)$. Nadarajah
módszerét használva a $k(x) = 1$, ha $-1/2 < x < 1/2$ (és 0 különben) magfüggvénnyel és a $h = 0.5$ ablakszélességel, mi lesz az $E(Y|X = 1.1)$ becslése?

(70) Vezesse le egyszerű hipotézisekre, hogy az egymintás u-próba valószínűséghányadospróba az egyoldali ellenhipotézis esetén!

(71) Hogyan tudjuk a lineáris modellt alkalmazni polinommal történő közelítésre?

(72) Adjon példát olyan maximum likelihood becslésre, mely nem torzítatlan.

(73) Definiálja az előjelpróbát! Mik a kritikus értékek? (3x)

Nemparaméteres próba, illeszkedésvizsgálat $\rightarrow$ adott eloszlású-e a minta?

Az előjelpróba a különbségek előjelén alapul. $H_0$ esetén Binom(n; 1/2).

(74) Adjon példát olyan $X_n$ valószínűségi változó-sorozatra, amely független tagú, de mégsem teljesül rá a centrális határeloszlástétel.

(75) Legyen X egyenletes eloszlású a [-1, 1] intervallumon. Számítsa ki $X^2$ sűrűségfüggvényét.

(76) Definiálja a normális eloszlást és vezesse le a legfontosabb tulajdonságait.

(77) Mondja ki és bizonyítsa is be a Markov egyenlotlegnséget.

(78) Hogyan alkalmazhatjuk a Kolmogorov-Szmirnov próbát véletlenszám-generátorok tesztelésére?

(79) Definiálja a binomiális eloszlást, vezesse le a képletét és adja meg alkalmazási lehetőségeit!

(80) Legyenek az adataink: 1,2,5,6. Számolja ki a mediánt és rajzolja fel a tapasztalati eloszlásfüggvényt!

(81) Tegyük fel, hogy egy 1ö fős csoportban mindenki kihúzza valakinek a nevét a karácsonyi ajándékozásnál. várható értékben hány ember húzta saját magát? Vezesse is le az eredményt!

(82) Rajzolja fel a $\lambda = 1$ és $\lambda = 2$ paraméterű exponenciális eloszlás sűrűség- és eloszlásfüggvényét!

(83) Vezesse le a $[0, \theta]$ intervallumon egyenletes eloszlásra a maximum likelihood becslést n mintaelem alapján!

(84) Rajzolja fel az alábbi minta tapasztalati eloszlásfüggvényét: 1,5,6,7! (2x)

(85) Hogyan tudunk inverz módszerrel véletlen számot generálni? (2x)

Legyen X val. vált., F eloszlásfüggvénnyel,
amely monoton növekedı és folytonos. Ekkor

\begin{itemize}
	\item F(X) egyenetes eloszlású [0, 1]-en.
	\item Ha $U \sim U(0, 1)$ akkor $F^{-1}(U)$ eloszlásfüggvénye F.
\end{itemize}

Pl.:

$$X \sim exp(\lambda) \rightarrow F(x) = 1-exp(-\lambda x) \rightarrow F^{-1}(x) = -\ln (1-x)/\lambda \rightarrow -\ln(1-U)/\lambda \sim exp(\lambda)$$

(86) Legyen az X valószínűségi változó sűrűségfüggvénye $f(x) = x^2/9$, ha $0 < x < 3$ és $0$ különben. Adjuk meg $X$ eloszlásfüggvényének értékét az $1/2$ helyen. $E(1/X)=?$ Adjuk meg $sqrt{X}$ sűrűségfüggvényének értékét az $1/2$ helyen.

(87) Számítsa ki annak a valószínűségét, hogy egy háromgyerekes családban azonos neműek a gyerekek! (Tegyük fel, hogy a fiúk és lányok szóletési valószínűsége is 0.5 és hogy az egyes születések függetlenek.)

(88) Definiálja a várható érték fogalmát diszkrét valószínűségi változókra!

A $p_i =P (X=x_i)$ eloszlással megadott valószínőségi változó várható értéke $E(X):= p_1x_1+ p_2x_2 +...,$ ha a sor abszolút konvergens.

(89)  Legyen X egyenletes eloszlású a $[-2,2]$ intervallumon. Adja meg $(X - 2)(X + 2)$ sűrűségfüggvényét.

(90) Mondja ki abszolút folytonos valószínűségi változók függetlenségének ekvivalens jellemzőit.

Ha X koordinátái függetlenek, akkor definíció szerint:

$F_X(z) = P(X_1<z_1, X_2<z_2, ..., X_d<z_d) = F_1(z_1)F_2(z_2)...F_n(z_n) \forall z \in \mathbb{R}^d$

(91) Mondjon példát az egyenletes, az exponenciális és a normális eloszlás alkalmazására!

Egyenletes eloszlás:\\
Véletlengenerátorokban igyekeznek az egyenletes eloszlást egy adott intervallumon minél inkább közelíteni

Exponenciális eloszlás:\\
Előrejelzéseknél alkalmazzák: n adatpontos minta megfigyelése után egy ismeretlen exponenciális eloszlású forrásból meg tudjuk jósolni, hogy mi lesz a következő érték ugyanabból az ismeretlen exponenciális eloszlású adatforrásból

Normális eloszlás:\\
Mérési hibákra, mért ingadozásokra és olyan élettartam vizsgálatokra használjuk, ahol a készülékek, alkatrészek rendszeres kopással mennek tönkre. Bizonyos feltételek eseték a binomiális eloszlás közelítésére is használható.

(92) Definiálja a bootstrap módszert.

\begin{itemize}
	\item Adott $X_n$-ből m elemű visszatevéses mintát veszünk (általában m=n) \quad $X^{*}_m = \{X^{*}_1,...,X^{*}_m\}$
	\item $X_i^{*}$ közös eloszlása: $F_n = n^{-1} \sum^n_{i=1} \delta_{x_i}$
	\item $T^{*}_{m,n} = t_m (X^{*}_m;F_n)$
	\item Ismétlés $\rightarrow \hat{G}_{m,n}$
	
\end{itemize}


(93) Mi a leíró statisztika feladata?

Célja egy már rendelkezésre álló, valóságra vonatkozó adathalmaz összefoglalása, elemzése, egyszóval az információtömörítés.\\

(94) Definiálja a kovarianciát!

$$cov(X, Y) = E(XY) - EXEY$$

(95) Mihez tart $P(X1+X2+ ... +Xn > 9n/2)$, ha $X1, X2, . . .$ független, a $[4,6]$ intervallumon
egyenletes eloszlású valószínűségi változók és n tart végtelenhez?

(96) Vezesse le az exponenciális eloszlás várható értékére vonatkozó képletet!

(97) Legyen az alábbi 3 megfigyelésünk az $(X, Y)$ párra. $(1;2)$ $(1.3;2.5)$ $(0.8;1.5)$. Nadarajah
módszerét használva a a $k(x) = 1$, ha $-1/2 < x < 1/2$ (és 0 különben) magfüggvénnyel és a $h = 0.5$ ablakszélességgel, adjon becslést $Y$ közelítésére, ha $X = 1$.

(98) Definiálja az egy-és kétmintás u-próbát.

(99) Az alábbi minta: 0,0,1,1,2 alapján becsüljük torzítatlanul az alábbi paramétereket:
várható érték, szórásnégyzet, annak a valószínűsége, hogy $X > 0.2$.

(100) Adjon módszert arra, hogyan szimulálna véletlen számot a $\sin x (0 < x < \pi/2)$ sűrűségfüggvényű eloszlásból!

(101) Hogy hívjuk a $h(t) = E(|X - t|^2)$ függvény minimum helyét?

(102) Hogyan adhatjuk meg a megszámlálható valószínűségi mezőn a valószínűséget?

(103) Legyen X 5 várható értékű és 2 szórású normális eloszlású, Y pedig az X = 7 esemény indikátora. Számítsa ki Y várható értékét és szórását!

(104)  Definiálja a visszatevés nélküli mintavétel modelljét és adja meg a különböző selejtszámok valószínűségeit!

(105) Mondja ki a Markov egyenlőtlenséget!

(106) Vezesse le a valószínűségre a maximum likelihood becslést n mintaelem alapján!

(107) Definiálja a sztochasztikus konvergenciát! (2x)

 minden $\mathcal{E}$, $\delta>0$-
hoz megadható olyan $n_0$, hogy $n>n_0$ esetén $P(|X_n-X| \geq \mathcal{E}) \leq \delta$.

Lemma

$$X_n \rightarrow X m.m. \leftrightarrow \lim_{m \rightarrow \infty} P(\sup_{n \geq m} |X_n - X| > \mathcal{E}) = 0$$

Köv.: A m.m. konvergenciából következik a
sztochasztikus konvergencia.


(108) Írja le a sűrűségfüggvény becslésére tanult eljárást.

(109) Adja meg a legkisebb négyzetes becslést a lineáris modellben. Mik a tulajdonságai?

(110) Legyen X egyenletes a $[0, 1]$-en. Adja meg $X^2$ eloszlását és várható értékét.

(111) Számolja ki a Wilcoxon próbastatisztika értékét az alábbi két adatsorra! Hogyan döntene arról, hogy tekinthetők-e azonos eloszlásból származónak? Az első minta: 1,2,4,12,23,50;
a második minta: 3,8,15,25,61,200.

(112) Definiálja az F-próbát és adja meg legfontosabb alkalmazásait. (2x)

$$H_0: \sigma_1 = \sigma_2$$
$$H_1: \sigma_1 \neq \sigma_2$$
Probastatisztika
$$F= \frac{(s_1^{*})^2}{(s_2^{*})^2} \stackrel{H_0 eseten}{\sim} F_{n-1,m-1}$$

(113)  Hogyan generálhatunk véletlen számot Neumann módszerével?

Legyen f(x) tetszőleges sűrűségfüggvény, $g(x)$ pedig olyan sűrűségfüggvény, amelyre $f(x) < Mg(x)$, valamely $M>1$ esetén és $g(x)$-ből könnyen tudunk mintát venni (tipikus példa az egyenletes eloszlás).

(114) Töltse ki az alábbi táblázat hiányzó celláit úgy, hogy a peremeloszlások egyenletesek
legyenek a megadott számokon! Függetlenek-e ezek a változók? Számolja ki az $E(X^2Y)$ várható értéket!

\begin{center}
 \begin{tabular}{| c c || c | c | c | c |} 
 \hline
 Y & X &  0 & 1 & 2 & 3\\ [0.5ex] 
 \hline\hline
 0 &  & 0,2 & & 0,05 & \\ 
 \hline
 1 &  & & 0,15 & & \\
 \hline\hline
\end{tabular}
\end{center}

(115) Mutassa meg, hogy független azonos paraméterű binomiálisok összege binomiális eloszlású! Mi a kapcsolat a rendek között?

(116) Definiálja két esemény függetlenségét!

Az A és B események függetlenek, ha $P(A \cap B) = P(A)P(B)$.

(117) Számítsa ki két kockadobás maximumának a várható értékét.

(118) Definiálja az első- és másodfajú hiba fogalmát, valamint próbák konzisztenciáját.

Első:\\
Valamely nullhipotézis statisztikai próbával végzett vizsgálatában az a hiba, amelyet a nullhipotézis elutasítása jelent amikor az igaz. Valószínûsége az elsõfajú kockázat (alfa). 

Másodfajú hiba\\
Az a hibás döntés, hogy nem vetjük el a helytelen nullhipotézist abban az esetben, amikor az alternatív hipotézis igaz. A próba közvetlenül nem ellenõrzi, ezért ha a próba nem utasítja el a nullhipotézist, ez még nem jelenti azt, hogy a nullhipotézist megfelelõ statisztikai biztonsággal elfogadhatjuk. Valószínûsége a másodfajú kockázat (béta). 



(119) Irja le a t-próba alkalmazását a párosított megfigyelések esetére.

(120) Vajon van-e stacionárius megoldása az $X_n = 0.96 X_{n-1} + \epsilon_n$ rekurziónak? Válaszát
indokolja!

(121) Legyen a mintaelemek sűrűségfüggvénye $f(x) = (\theta + 1)x^{\theta}$ ha $0 < x < 1$ ($\theta \geq$ 0 a par.).
Egy mintaelem alapján határozzuk meg az $\alpha = 0.05$ terjedelmű valószínűséghányadospróbát a $H_0 : \theta = 0, H_1 : \theta = 1$ hipotézisekre!

(122) Ismertesse a lineáris regresszió együtthatóira vonatkozó próbák lényegét! (2x)

??

(123) Adott n elemű minta az alábbi eloszlásból: $P(Z = 1) = c$, $P(Z = 2) = 2c$, $P(Z = 3) = 1 - 3c$ ($0 < c < 1/3$ az ismeretlen paraméter). Határozza meg a paraméter maximum likelihood becslését!

(124) Számítsa ki két kockadobás minimumának a várható értékét.

(125) Írja fel a Csebisev egyenlőtlenséget, és vázolja a bizonyítását.

Legyen X tetszőleges valószínűségi változó, melyre $D^2X < \infty$ és $\mathcal{E} > 0$ tetszőleges. Ekkor

$$P(|X - EX| \geq \mathcal{E}) \leq \frac{D^X}{\mathcal{E}^2}$$

Bizonyítás:

???

(126) Legyen X 5 várható értékű Poisson eloszlású, Y pedig az X = 5 esemény indikátora. Számítsa ki Y várható értékét és szórását!

(127) Adjon példát olyan maximum likelihood becslésre, mely nem torzítatlan.

pl. a normális eloszlás szórásnégyzetére kapott becslés ilyen

(128) Adja meg a Wilcoxon-próba hipotéziseit, a próbastatisztikát és a kritikus tartományt.
Mikor alkalmazná ezt a próbát? (2x)

???

(129) Adjon meg két módszert integrálok szimulációval történő kiszámítására! Hasonlítsa is
össze őket!

\begin{itemize}
\item Integrálok közelítése
\item Bootstrap
\end{itemize}

Az első eljárás is torzítatlan és szórásnégyzete $(E(g^2(X)) - I^2) / N$, amiből kisebb szórás adódik.\\
Tovább is javítható, ha X az [a,b]-ra koncentrálódó f sűrűségfüggvényű eloszlásból származik:
$$\int^b_a g(x)dx = \int^a_b \frac{g(x)}{f(x)} f(x)dx = E (\frac{g(X)}{f(X)})$$
és ennek még kisebb a szórása, ha $f(X) \approx g(X)$\\
Tehát g(X)/f(X) szimulált értékeinek átlaga jó közelítés\\
Az is előnye, hogy improprius integrálokra is alkalmazható (ha pl. f a normális eloszlás)


(130) Definiálja a 2 mintás t-próbát és vezesse le a próbastatisztika képletét!

$$t = \sqrt{\frac{nm}{n+m}}\frac{\overline{X} - \overline{Y}}{\sqrt{\frac{(n-1)(s^*_1)^2 + (m-1)(s^*_2)^2}{n+m-2}}} \stackrel{H_0 eseten}{\sim}
t_{n+m-2}$$

(131) -

(132) Vezesse le a normális eloszlás várható értékére a maximum likelihood becslést n mintaelem alapján!

$$L(\mu, \sigma^2;x) 
= \prod^n_{k=1} \frac{1}{\sqrt{2\pi \sigma^2}} 
e^{\frac{(x_i-\mu)^2}{2\sigma^2}}
=\left(\frac{1}{\sqrt{2\pi \sigma^2}}\right)^n 
e^{-\frac{1}{2 \sigma^2}} \sum^n_{k=1} (x_i - \mu )^2$$

$$\ln L(\mu, \sigma^2;x) = - \frac{n}{2} \ln 2\pi - \frac{n}{2} \ln \sigma^2 - \frac{1}{2\sigma^2} \sum^n_{k=1} (x_i - \mu)^2$$
$$(\ln L(\mu,\sigma^2;x))^{'}_{\mu} = - \frac{1}{2\sigma^2}(-2)\sum^n_{k=1}(x_i - \mu)$$

Átrendezve $(\ln L(\mu,\sigma^2;x))^{'}_{\mu} = 0$ egyenletet, kapjuk, hogy $\displaystyle{\hat{\mu} = \frac{\sum^n_{k=1} X_i}{n} = \overline{X}}$. Ez valóban maximum, mivel $(\ln L(\mu,\sigma^2;x))^{''}_{\mu} = -\frac{n}{\sigma^2} < 0$

(133) Írja fel, az X diszkrét valószínűségi változó várható értékének definícióját és sorolja fel
a várható érték legfontosabb tulajdonságait!

Egy X valószínűségi változó várható értéke

\begin{itemize}
	\item diszkrét esetben
		$$E(X) := \sum_k x_k P(X = x_k) = \sum_k x_k p_k$$
		
	\item folytonos esetben
		$$E(X) := \int_{-\infty}^\infty xp(x)dx$$
\end{itemize}

Nem minden valószínűségi változónak van véges várható értéke\\
Ha $E(X)$ véges, akkor az abszolút konvergencia miatt egyértelmű is.\\
Ha $EX$ véges, akkor $E(aX+b)=aEX+b$\\
Ha $EX$ és $EY$ véges, akkor $E(X+Y)=EX+EY$

(134) Legyen $\Omega = {1, 2, 3, 4, 5, 6}$, $P(i) = 1/6 (1 \leq i \leq 6)$ és $A = {1, 2, 3}$. Adon meg olyan B
eseményt, amire $0 < P(B) < 1$ és B és A függetlenek! Válaszát indokolja.

 Például $B = \{3, 4\}$, mert ekkor $P(A \cap B) = 1/6$ és $P(A)P(B) = \frac{1}{2} * \frac{1}{3}$, tehát egyenlőek, azaz a két esemény független.

(135) Tegyük fel, hogy az X valószínűségi változó sűrűségfüggvénye $f(x) = 2(1-x)$ ha $0 < x < 1$ és 0 különben. E(X) =? 

$$EX = \int^1_0 xf(x)dx = ... = 1/3$$

(136) Legyen az X valószínűségi változó eloszlása $P(X = k) = (5 - k)/15$, $ha 0 \leq k \leq 4$. $D^2(X)$ = ?

$D^2(X) = E(X^2) - E^2(X)$,$E(X) = 1 * \frac{4}{15} + 2 * \frac{3}{15} + 3 * \frac{2}{15} + 4 * \frac{1}{15} = \frac{20}{15}$.\\
$E(X^2) = 1 * \frac{4}{15} + 4 * \frac{3}{15} + 9 * \frac{2}{15} + 16 * \frac{1}{15}$.

(137) Mondja ki a nagy számok Bernoulli-féle törvényét!

A nagy számok törvényének legelső verzióját
még Bernoulli bizonyította,
indikátorváltozókra: eszerint azonos
körülmények között elvégzett független
kísérleteknél tetszőleges esemény relatív
gyakorisága tart az esemény
valószínűségéhez.

(138) Definiálja egy valószínűségi változó mediánját.

Az x valószínűségi változó mediánját $\overline{x}$ vagy $\mu_{1/2}(x)$ jelöli.
Páratlan elemszám esetén a középső elem, páros elemszám esetén a két középső elem átalaga.

(139) Mi a lényeges különbség a leíró és a matematikai statisztika között?

Leíró:\\
Célja egy már rendelkezésre álló, valóságra vonatkozó adathalmaz összefoglalása, elemzése, egyszóval az információtömörítés.\\
Matematikai:\\
Célja a megfelelő – vagyis a sokaság egészének paramétereit legjobban tükröző, reprezentáló – minta kiválasztása, a sokasági paramétereknek a minta paramétereivel történő becslése, illetve a sokasági paraméterekre vonatkozó feltételezések, hipotézisek elfogadása vagy elvetése. Foglalkozik továbbá a valóság összefüggéseinek egyszerűsített megragadására törekvő modellekkel is, mint az idősor- és regressziós modellek.

(140) Vezesse le a maximum likelihood becslést a binomiális eloszlás p paraméterére (az n paramétert tekintsük ismertnek)!

$$L(m, p; x) = \prod_{k=1}^n \displaystyle{\left({m \atop x_k}\right)}p^{x_k}(1-p)^{m-x_k} \quad (x_k = 0, 1, ..., m)$$
$$\ln L(m, p; x) = \sum^n_{k=1} \ln \left ({m \atop x_k}\right) + \ln p \sum^n_{k=1} x_k + \ln(1-p) \sum^n_{k=1}(m-x_k)$$
$$(\ln L(m,p;x))^{'}_p = \frac{1}{p} \sum^n_{k=1}x_k + \frac{1}{1-p} \sum^n_{k=1} (m - x_k) = \frac{1}{p} \sum^n_{k=1}x_k + \frac{1}{1-p}
\left(nm- \sum^n_{k=1}x_k \right) = \frac{1}{p}n\overline{x} + \frac{1}{1-p}(nm-n\overline{x})$$

Átrendezve a $(\ln L(m,p;x))^{'}_p = 0$ egyenletet, kapjuk, hogy $\hat{p} = \frac{\overline{X}}{m}$.  Ez valóban maximum, mivel $(\ln L(m,p;x))^{''}_p$-t kiértékelve a $\hat{p}$ helyen $(\ln L(m,p;x))^{''}_p = \frac{-n \overline{x}}{p^2} + \frac{-n(m-\overline{x})}{(1-p)^2}=
-n\left(\frac{\overline{x}}{p^2}+\frac{m - \overline{x}}{(1-p)^2}\right)<0$.

(141) Definiálja a hipotézisvizsgálat alapfogalmait!

Hipotézis: állítás, aminek igazságát vizsgálni szeretnénk
Statisztikai próba: eljárás, aminek a segítségével döntést hozhatunk a hipotézisről
Nullhipotézis: $H_0 : \theta \in \Theta$\\
Ellenhipotézis: $H_1 : \theta \in \Theta$\\
Paramétertér: $\Theta = \Theta_0 \cup \Theta_1$\\
Mintater: $\chi = \chi_e \cup \chi_k$\\
$\chi_k$: kritikus tartomány - azon X megfigyelések halmaza, amikre elutasítjuk a nullhipotézist\\
$\chi_e$: elfogadási tartomány - azon X megfigyelések halmaza, amikre elfogadjuk a nullhipotézist\\
A $\chi$ mintatér felosztását egy $T : X \rightarrow R$ statisztika (neve: próbastatisztika) segítségével végezzük el\\
P(elsőfajú hiba) = $\alpha(\theta) = P_\theta$ (elvetjük $H_0$-t $|$ $H_0$ igaz)\\
Terjedelem: $\alpha = sup\{\alpha(\theta) : \theta \in \Theta_0\}$\\
P(másodfajú hiba) = $\beta(\theta) = P_\theta$ (elfogadjuk $H_0$-t $|$ $H_0$ hamis)\\
Erőfüggvény: $\Psi(\theta) = 1 - \beta(\theta) = P_\theta$ (elvetjük $H_0$-t $|$ $H_0$ hamis)

(142)  Definiálja az u-próbát az egy- és kétmintás esetre is.

Egymintas

$$T(X) = u = \frac{\overline{X} - m_0}{\frac{\sigma}{\sqrt{n}}} \stackrel{H_0 eseten}{\sim} N(0, 1)$$

Kritikus tartomany

$$\chi_k = \{X: |u| > u_{1-\frac{\sigma}{2}}\}$$
$$\chi_k = \{X: u > u_{1-\alpha}\}$$
$$\chi_k = \{X: u < u_\alpha\}$$

Ketmintas

$$\displaystyle{u = \frac{\overline{X} - \overline{Y}}{\sqrt{\frac{\sigma_1^2}{n} + \frac{\sigma_2^2}{m}}}}$$

(143) Mi a hasonlóság és mi a különbség az előjel- és a Wilcoxon próba között?

EZ MI A FASZ?

(144) Adjon módszert arra, hogyan szimulálna véletlen számot a sin(x) $(0 < x < \pi/2)$ sűrűségfüggvényű eloszlásból!

??

(145) Definiálja a korrelogrammot és adjon módszert, hogyan lehet ennek segítségével tesztelni
az autokorrelációk szignifikanciáját.

A korrelogram a korrelációs statisztikát megjelenítő diagram. Az idősor analízis, másnéven az autokorrelációs diagram az autokorreláció $r_h$ és a $h$ (timelag) megjelenítése.

A statisztikai próba hipotétisei:

$$H_0 : p_k = 0$$
\[
	\text{és}
\]
$$H_1 : p_k \neq 0$$

Könnyen ellenőrizhető, hogy a normális eloszlással való közelítés megfelelő
lesz, így a próbastatisztika

$$\frac{\hat{p}_k}{1/\sqrt{T}} \sim N(0,1) $$

Elfogadási tartomány:

$$[-z_{1-\alpha/2},z_{1-\alpha/2}]$$

A korrelogram ábrájáról könnyen leolvasható, hogy a kapott autokorrelációs
értékek ebbe a sávba esnek-e, vagy sem.

(146) Definiálja valószínűségi változók általános fogalmát.

$X : \Omega \rightarrow R$ függvény valószínűségi változó, ha $\{\omega: X(\omega) \in B\} \in \mathscr{A}$ minden $B$ Borel halmazra $(X: \Omega \rightarrow R$ Borel mérhető függvény).
Ha $\mathscr{A} = \mathscr{P}(\Omega)$, akkor minden $\Omega \rightarrow R$ függvény valószínűségi változó.


(147) Definiálja n darab esemény függetlenségét.

Az $A_1,A_2, ..., A_n$ eseményeket teljesen függetleneknek (röviden függetleneknek) nevezünk, ha az ${1, 2, ..., n}$ index halmaz bármely ${i, j, ..., k}$ részhalmazára

$$P(A_i \cap A_j \cap ... \cap A_k) = P(A_i)P(A_j)...P(A_k)$$

(148) Mutassa meg, hogy a peremeloszlások nem határozzák meg az együttest!

Például legyen X egy érmedobás eredménye, Y pedig egy másik, független érmedobás eredménye. Ekkor (X, Y) peremeloszlásai ugyanazok mint (X, X) peremeloszlásai, de az együttes eloszlások különböznek.

(149) Adjon példát korrelálatlan de nem független valószínűségi változó-párra!

Például a 148. példabeli X, Y -ra X + Y és X - Y ilyen.

(150) Legyen az X eloszlása a következő: $P(X = k) = k/6$, ha $k = 1, 2, 3$. Számolja ki $E(X)$ és $E(X^2)$ értékét!

$E(X) = 7/3$, $E(X^2) = 6$.

(151) Mondja ki a centrális határeloszlás tételt! Hol használtuk a statisztikában?

Legyenek $X_1 , X_2 ,..., X_n ,...$ független, azonos eloszlású valószínűségi változók. Tegyük fel, hogy $D^2(X) = \sigma^2 < \infty$ véges $(m:=E(Xi))$. Tekintsük a standardizált összegüket:

$$Z_n = \frac{X_1 + ... + X_n -nm}{\sqrt{n}\sigma}$$

Ekkor $Z_n$ gyengén konvergál a standard normális eloszláshoz,
azaz

$$P(\frac{X_1 + ... + X_n - nm}{\sqrt{n}\sigma} < z) \rightarrow \Phi(z)$$

ahol $\Phi$ a standard normális eloszlás eloszlásfüggvénye.

(152) Hasonlítsa össze a várható érték tesztelésre vonatkozó próbákat!

Az u és a t próba is normális eloszlású mintákra alkalmazható, mindegyik konzisztens és torzítatlan, az egyoldali próbák legerősebb próbák is. Ha $n \rightarrow \infty$ akkor a tpróba átmegy az u próbába

(153) Tegyük fel, hogy a mintánk megfigyelt értékei: 0, 0, 2, 2 és azt is, hogy közelíthetőek 1 szórású normális eloszlással. Írjon fel 0.95 megbízhatóságú konfidencia intervallumot a várható értékre!

(154) Definiálja a $\chi$-négyzet próbát és adja meg legfontosabb alkalmazásait.

H0 hipotézis: az $A_1, A_2, ..., A_r$ teljes eseményrendszerre teljesül $P(A_1)=p_1$, $P(A_2)=p_2$, ...,$P(A_r)=p_r$.\\
Tesztstatisztika: $\displaystyle{ \sum^r_{i=1} \frac{(v_i - np_i)^2}{np_i}}$\\
ami aszimptotikusan r-1 szabadságfokú $\chi$-négyzet eloszlású, ha igaz a nullhipotézis. 

\begin{itemize}
	\item Diszkrét illeszkedésvizsgálat
	\item Függetlenségvizsgálat
	\item Homogenitásvizsgála
	\item Egyszerű lineáris regresszió
\end{itemize}

(155) Egy faktor hatását vizsgáljuk. Az alacsony szinten 4 mérésünk volt: 8, 7, 11, 10. A felső
szinten az eredmények: 5,7,3,9. Hogyan vizsgálnánk a hatás szignifikanciáját? Adjuk meg
a próbastatisztika értékét.

(156) Melyik konstans közelíti az $X$ változót a legkisebb négyzetes hibával? Válaszát indokolja!

$$??$$

(157)  Irja le a Monte Carlo módszerek lényegét és adjon példát alkalmazásukra

\begin{itemize}
	\item Véletlen szám generátorok
	\item Szimulációk
	\subitem Integrálok közelítése
	\subitem Bootstrap
	\item Markov láncokon alapuló algoritmusok
\end{itemize}

Példa: LCG: $X_{n+1} = (aX_n + c)$ mod m

(158) Definiálja a gyengén stacionárius idősorokat és adjon rá egy példát!

Gyenge stacionaritás: a kovariancia-struktúra állandó

Az ($Y_t$) idősor gyenge értelemben stacionárius, ha első- és második momentuma eltolásinvariáns, azaz $EY_t = m$ minden $t$ esetén, és $\gamma(t,s) = Cov(Y_t, Y_s) = \gamma(t - s)$ bármely t,s pár esetén.

(159) Legyen az X valószínűségi változó sűrűségfüggvénye $f(x) = 2(3 - x)/9$ ha $0 < x < 3$ (és 0 különben). $E(X)$ =?, $D^2(X)$ =? 

(160) Legyen X a pikkek, Y pedig a királyok száma annál a kísérletnél, ahol kétszer húzunk
visszatevéssel az 52 lapos francia kártyacsomagból (ebben 4 szín van és minden színből 13
lap). Írja fel X, Y együttes eloszlását!

(161) Definiálja a feltételes valószínűséget!

$$P(A|B) = \frac{P(A \cap B)}{P(B)}$$

ha $P(B) \neq 0$

(162) Mondja ki a teljes valószínűség tételét!

Legyen B1, B2, ..., pozitív valószínűségű eseményekből álló teljes eseményrendszer, A $\in$ $\mathscr{A}$ tetszőleges. Ekkor
$$P(A) = P(A|B_1)P(B1)+P(A|B_2)P(B2)+ ...$$


(163) Irja le a Parzen-Rosenblatt módszert és tulajdonságait.

Tapasztalati eloszlásfüggvény nem deriválható, de ezen segíthetünk, ha az egyes megfigyeléseket nem pontszerűnek, hanem az adott érték körüli kicsi szórású folytonos eloszlásúnak képzeljük (ez az eloszlás a magfüggvény).\\
Ennek a folytonos keverékeloszlásnak a deriváltja jól közelíti a sűrűségfüggvényt.\\
Tétel. Ha a mintánk egy $f(x)$ sűrűségfüggvényű
eloszlásból származik, a $k(y)$ magfüggvény
egyenletesen korlátos és $yk(y)$ határértéke a
végtelenben 0, valamint $h_n$ olyan számsorozat,
melyre $\lim h_n = 0$ és $\lim nh_n = \infty$, akkor
$$f_n(x) = \frac{1}{nh_n} \sum^{n}_{i=1}k\left(\frac{x-X_i}{h_n}\right)$$ 
aszimptotikusan torzítatlan, konzisztens becslés az f(x) minden folytonossági pontjában.

(164) Adja meg valószínűségi változók függetlenségének karakterizációit!

Ha \underline{X} koordinátái függetlenek, akkor definíció szerint\\
$F_X(z)=P(X_1<z_1, X_2< z_2,..., X_d<z_d)=F_1(z_1)F_2(z_2)...F_d(z_d)$ (minden z $\in$ $R^d$–re).\\
Meg is fordítható: F szorzatelőállításából következik a függetlenség\\
Deriválva: a függetlenség abszolút folytonos változókra ekvivalens a sűrűségfüggvény 
$f_X(z)=f_1(z_1)f_2(z_2)...f_d(z_d)$ alakú előállításával is.

(165) Definiálja a QQ plot diagramot! Mikor használjuk?

A megfigyelt és az illesztett eloszlás kétdimenziós ábrázolása. Eloszlásfüggvény q-kvantilise: az az érték, amelynél q valószínűséggel kapunk kisebbet: $G^{-1}(q)$.\\
Eloszlás illeszkedésének vizsgálatára használjuk.

(166) Mit jelent a becsléses illeszkedésvizsgálat és hogyan alkalmazzuk rá a $\chi$-négyzet próbát?

Legyen $\theta$ egy $s$ dimenziós paramétervektor, valamint legyen $\hat{\theta}$ a $\theta$ paramétervektor MLbecslése, és legyen $\hat{p}_j$ = $p_j (\hat{\theta}$)\\
$H_0$ : $P(X_i = x_j)$ = $\hat{p}_j$ $j = 1,. . . , r$\\
$H_1$ : $\exists$ legalább egy $j$ melyre $P(X_i = x_j)$ $\neq$ $\hat{p}_j$\\
Probastatisztika: $T_n = \displaystyle{\sum_{j=1}^{r}}\frac{(v_j - n\hat{p}_j)^2}{n\hat{p}_j}$ $\stackrel{H_0 eseten}{\sim}$ $\chi^2_{r-s-1}$\\
Kritikus tartomány: $\chi_k$ = $\{x: T_n(x) > \chi^2_{r-s-1,1-\alpha}\}$\\
$T_n = n * \frac{(v_{11}v_{22} - v_{12}v_{21})^2}{v_1 * v_2 * v_1 * v_2}$ $\stackrel{H_0 eseten}{\sim}$ $\chi^2_1$.



(167) Milyen modern módszerekkel tudjuk a statisztikai eljárásokat a big data esetére alkalmazni?

\begin{itemize}
\item Részek elemzése, összesítése
\item Véletlen részminták vizsgálata
\item Mesterséges intelligencia
\end{itemize}


(168) Irja le a tanult, Neumann-féle módszert véletlen számok generálására!\\
Legyen f(x) tetszőleges sűrűségfüggvény, $g(x)$ pedig olyan sűrűségfüggvény, amelyre $f(x) < Mg(x)$, valamely $M>1$ esetén és $g(x)$-ből könnyen tudunk mintát venni (tipikus példa az egyenletes eloszlás).

(169) Mi a szóráselemzés lényege?\\
Egy vagy több faktor különböző „szintjein” mérünk eredményeket (pl. termésátlagokat)


(170) Mit vizsgáltunk? Értékelje a kapott eredményeket! Mi az "Intercept" és a "t" szemléletes
jelentése? Egészítse ki a táblázatot a hiányzó értékekkel!

\begin{verbatim}

	homb<-read.table("e:\\oktatas\\hom\\nyir-51-88m.hom")
	r=c(0:37); atl<-rep(0,times=30)
	for (i in 1:30) atl[i]=mean(homb[365*r+243+i,1])
	t<-c(1:30);lm1=lm(atl~t);summary(lm1)
	
\end{verbatim}
\begin{verbatim}
Az eredmeny:
Residuals:
Min 1Q Median 3Q Max
-0.78002 -0.23031 -0.01675 0.19913 0.78953
Coefficients:
Estimate Std. Error t value Pr(>|t|)
(Intercept) 18.682815 0.150876 ...... <2e-16 ***
t -0.197392 0.008499 -23.23 ..........
Residual standard error: 0.4029 on 28 degrees of freedom
Multiple R-squared: 0.9507, Adjusted R-squared: 0.9489
F-statistic: 539.5 on 1 and 28 DF, p-value: .....
\end{verbatim}

Minden ilyen feladatnál a statisztikai tartalom a lényeg. Azaz itt az lm függvény: ez
lineáris regressziót számol, melyben az atl (szeptemberben a napi középhőmérsékletek) értékét közelíti a t (nap sorszáma) lineáris függvényével. A modell jónak tűnik (R
2 = 0.95). Az
intercept a konstans tag (a szept 0-ára becsült átlaghő mérséklet), a t az egyes együtthatók
szignifikanciáját vizsgáló t-próba statisztika értéke. A pontok helyére rendre $18.68/0.15, < 2e - 16 ***,
< 2e - 16$ irandó



\end{document}